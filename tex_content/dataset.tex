\section{The dataset} \label{dataset}
The dataset that is studied is derived from videos of track inspection vehicles recorded during regular
inspections.
The video system of the SDS vehicle records both rails from two angles resulting in four video footage
parallel.
A single footage was selected as it provides a static positioning relative to the tracks with good
protection against changes of the lightning of the surroundings.
The video system records with a resolution of 720x288 (width x height) with RGB channels
at 50 fps rate.
The video files are binary files without any compression applied on them.

A single footage sample video of approximately 3 minutes was taken as a starting point.
This video contains a side view of a rail, that is defined as \emph{normal} rail
along with a few seconds of rail covered with \emph{grass} or showing a \emph{double rail} section.
Latter two is considered as outlier to conclude on the first experiment building up the anomaly
detector model.
These are not comparable to the real life rail defects, however they allow an easy model setup and
evaluation.
Some examples of the images extracted from the video is shown on Figure \ref{fig:example_images}.

\begin{figure}[H]
    \centering
    \begin{subfigure}{0.3\textwidth}
        \centering
        \includegraphics[width=\textwidth]{./data/sd1_sample/normal/img_00006.jpg}
        \caption*{Normal rail}
    \end{subfigure}
    \begin{subfigure}{0.3\textwidth}
        \centering
        \includegraphics[width=\textwidth]{./data/sd1_sample/normal/img_00723.jpg}
        \caption*{Normal rail}
    \end{subfigure}
    \begin{subfigure}{0.3\textwidth}
        \centering
        \includegraphics[width=\textwidth]{./data/sd1_sample/normal/img_04857.jpg}
        \caption*{Normal rail}
    \end{subfigure}
    \begin{subfigure}{0.3\textwidth}
        \centering
        \includegraphics[width=\textwidth]{./data/sd1_sample/grass/img_05649.jpg}
        \caption*{Rails covered with grass}
    \end{subfigure}
    \begin{subfigure}{0.3\textwidth}
        \centering
        \includegraphics[width=\textwidth]{./data/sd1_sample/double_rail/img_05676.jpg}
        \caption*{Double rails}
    \end{subfigure}
    \caption{Sample images from the dataset}
    \label{fig:example_images}
\end{figure}

This sample video was sliced to images, resulting in the dataset shown in Table \ref{table:dataset}.
The resulting dataset is imbalanced, and the so-called outliers can be easily identified.
In general annotation can not be assumed for such problems, however in this case it was
provided manually to grant the possibility of performance evaluation.

\begin{table}[H]
    \centering
    \begin{tabular}{l c}
        Image type              & Number of images \\
        \hline
        Normal rail             & 8640             \\
        Rail covered with grass & 64               \\
        Double rails            & 29               \\
        \hline
        Total                   & 8733             \\
    \end{tabular}
    \caption{Dataset obtained from sample video}
    \label{table:dataset}
\end{table}

Observing the quality of captured video frames more in detail we can see that
there is no slurring on the images that is remarkable considering that the recording was taken
with a vehicle speed up to even 80 to 100 kilometers per hour.
However, a slight fish eye distortion can be seen that is noticeable mostly in the rail itself,
as it is not tend to follow a straight line, a small bending effect is given.
Also, the lighting of the pictures result in a brighter spot in the center.
The image quality remains stable in the sample video, including the illuminance that is
secured by the shrouds applied on the vehicle around the cameras.

Later on the course of the work MÁV CRTI Ltd. provided further videos from both vehicles
ranging up to 450 GB of raw data that can be used for further training, evaluation or optimization
the models.
In these videos 68 rail defects are recorded, and their approximate location is also known.
These defects are hidden in recordings ranging up to a duration of some hundred hours.
This already indicates the main challenge of this problem, that failure rate (faulty images relative
to all images) is well below the value of $1\%$.