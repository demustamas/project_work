\section{The dataset} \label{dataset}
The video system of the SDS vehicle records both rails from two angles resulting in four video footages
parallel.
A single footage was selected as it provides a static positioning relative to the tracks with good
protection against changes of the lightning of the surroundings.
The video system records with a resolution of 720x288 (width x height) with RGB channels
at 50 fps rate.
Some examples of the images extracted from the video is shown on Figure \ref{fig:example_images}.

A single footage sample video of approximately 3 minutes was provided as a starting point.
This video contains a side view of a rail, that is defined as \emph{normal} rail
along with a few seconds of rail covered with grass of showing a double rail section.

\begin{figure}[!ht]
    \centering
    \begin{subfigure}{0.3\textwidth}
        \centering
        \includegraphics[width=\textwidth]{./data/sd1_sample/normal/img_00006.jpg}
        \caption*{Normal rail}
    \end{subfigure}
    \begin{subfigure}{0.3\textwidth}
        \centering
        \includegraphics[width=\textwidth]{./data/sd1_sample/normal/img_00723.jpg}
        \caption*{Normal rail}
    \end{subfigure}
    \begin{subfigure}{0.3\textwidth}
        \centering
        \includegraphics[width=\textwidth]{./data/sd1_sample/normal/img_04857.jpg}
        \caption*{Normal rail}
    \end{subfigure}
    \begin{subfigure}{0.3\textwidth}
        \centering
        \includegraphics[width=\textwidth]{./data/sd1_sample/grass/img_05649.jpg}
        \caption*{Rails covered with grass}
    \end{subfigure}
    \begin{subfigure}{0.3\textwidth}
        \centering
        \includegraphics[width=\textwidth]{./data/sd1_sample/double_rail/img_05676.jpg}
        \caption*{Double rails}
    \end{subfigure}
    \caption{Sample images from the dataset}
    \label{fig:example_images}
\end{figure}

This sample video was sliced to images, resulting in the dataset shown in Table \ref{table:dataset}.
The resulting dataset is imbalanced, and the so-called outliers can be easily identified.
In general annotation can not be assumed for such problems, however in this case it was
provided manually to grant performance evaluation possibility.

\begin{table}[!ht]
    \centering
    \begin{tabular}{l c}
        Image type              & Number of images \\
        \hline
        Normal rail             & 8640             \\
        Rail covered with grass & 64               \\
        Double rails            & 29               \\
        \hline
        Total                   & 8733             \\
    \end{tabular}
    \caption{Dataset obtained from sample video}
    \label{table:dataset}
\end{table}

There is no slurring observed on the images that is remarkable considering that the video is taken
with a vehicle speed up to 80 to 100 kilometer per hour.
However a slight fisheye distortion can be seen that is noticeable mostly in the rail itself,
as it is not tend to follow a stright line, a small bending effect is given.
Also the lighting of the pictures result in a brighter spot on the center.
The image quality remains stable in the sample video, including the illuminance that is
secured by the shrouds applied on the vehicle around the cameras.

During modeling besides augmentation no image manipulations were applied
except the normalization during entering the neural network
and resizing to 704 x 288 x 3 to ensure decoding of the images to the same original size
(please see Section \ref{model}).

Later on the course of the work MÁV CRTI Ltd. provided further videos from both vehicles
ranging up to 450 GB of raw data that can be used for further training, evaluating or tuning
the models.