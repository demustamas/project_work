\begin{abstract}
    Following project is a continuation of the Mathematic Modeling Practice subject.
    During the first semester a set of convolutional networks were built to compare
    on the task of classifying different railway track faults.
    The project was followed up by contacting MÁV Central Rail And Track Inspection Ltd.
    who has provided sample dataset for further research and study purposes.
    This dataset is limited in terms of track failures however it provides the opportunity
    to apply anomaly detection models.
    The sample contains video footage of a short section of a single track of approx. 3 minutes,
    with a few seconds of rail sections covered with grass and/or containing double tracks.
    The latter two is considered as outlier from the dataset.
    A set of autoencoder models built to detect these outliers in the sample.
    The autoencoder is based on the convolutional models of VGG19, ResNet50 and EfficientNetV2L.
    Different anomaly detection methods were applied, an approach based on the calculation of a loss
    measure between the input and the output of the autoencoder and IsolationForest algorithm
    applied on the feature space of the inputs generated by the encoder part.
\end{abstract}