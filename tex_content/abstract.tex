\begin{abstract}
    Railway track monitoring is an essential part of infrastructure maintenance since the beginning
    to ensure safety of railway operation.
    Several track faults are identified throughout the time, starting from minor surface defects
    up to major rail cracks and more severe damages.
    During normal railway operation the task of the maintenance personnel is to detect, identify
    and monitor the propagation of such defects, so when it comes to a certain degradation
    the necessary prevention steps can be taken to ensure railway safety in a sustainable level.
    Among many, a key inspection method is still visual inspection due to the high variety
    of possible faults and many aspects that need to be considered during evaluation of necessary steps.
    So far this has not been fully replaced with instrumented monitoring or any highly accurate
    approach that is prevalent in rail industry.
    Current study aims to develop such method by utilizing machine learning algorithms that went
    through severe development in the past decades.
    The actual case study relies on real life data received from MÁV Central Rail And Track
    Inspection Ltd., the responsible company for rail monitoring in Hungary.
    The dataset contains video recordings of the rails done by the measurement vehicles during regular
    track investigations.
    These video files are subjected to a first experiment, application of an autoencoder model
    followed by anomaly detection methods to detect frames of the video that might contain rail faults.
    The anomaly detection was performed by applying two methods, the Isolation Forest algorithm
    and a loss-based approach over a set of autoencoders with four different type of encoders.
    The project is limited to a first trial of a sample video file to gain first experiences
    with such models and to perceive application possibilities.
    In this dataset the outliers were defined as rails covered with grass and video frames containing
    double rail sections, and were successfully identified by most of the algorithms.
    The internal behavior of the model is visualized and examined by a subsequent PCA and t-SNE
    methods to reduce the dimensionality of the dataset.
    Current work is a continuation of the Mathematic Modeling Practice subject as part of a training
    held by the AI Research Group of Eötvös Lóránd University and considered as thesis work.
\end{abstract}