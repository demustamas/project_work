\section{Conclusion} \label{conclusion}
In current study an application of an Autoencoder as anomaly detector is presented.
The model is trained and evaluated of real life sample data provided by MÁV CRTI Ltd.
Four different models were built depending on the applied Encoder.
The anomaly detection is performed based on the loss values defined between input
and output images and based on the Isolation Forest algorithm applied to the bottleneck
vectors.
Performance evaluation covered visual check of the outliers, evaluation of classification
metrics, that was granted via manual annotation of the sample data.
The performance of the model is visualized via application of PCA and t-SNE algorithms that
provided an insight view how features are translated through the models.

The results highlighted that such models can be very effective on finding outliers
in a given dataset, however the following remarks have to be noted for the sake of completeness:

\begin{itemize}
      \item The sample dataset is limited in terms of variation, the video footage shows
            mostly the same side view of a \emph{normal} rail with two type of outliers:
            \emph{grass covered} and presence of \emph{double rails} which are very seldom found.
      \item The outliers differ from the normal images significantly and are concentrated to
            three main clusters in the dataset.
      \item There was no preprocessing of the images except resizing and normalization.
      \item The training of the neural networks is limited, there was no hyperparameter
            optimization done.
      \item The anomaly detection methods were also not optimized for the dataset.
\end{itemize}

Even though of these limitations the results of the models are remarkable, especially the
loss-based method provided good results: an indication of the three main outlier clusters.
In current use case it is vital to identify as many outliers as possible to ensure railway
operation safety.
On the other hand it is economically reasonable to limit the false positives to a minimum
to ensure efficient review of the results.
However, in real life application the task is not to identify grass near the rails or double rail
construction, much finer outliers need to be detected: surface issues of the rail running surface
or missing parts of the rail fastening.
This highlights the possible limitations of the current model.
On the other hand this draws the attention to additional use cases coupled to this challenge.
The model might be able to classify the images to main rail construction groups, for example
separate the images to normal rails, turnouts, bridges, ballast covered track, etc.
Later on a more refined model can be used on these groups for anomaly detection.

The hope is not yet lost due to the many options present to improve our model.
A shortlist of these improvements is listed below:
\begin{itemize}
      \item Apply image preprocessing, like CLAHE histogram equalization to boost the learning
            capability of the neural networks.
      \item Optimize parameters of the neural network to fit to given problem and dataset.
      \item Besides classification, segment the images and detect major elements of the track,
            such as rails, fastening elements to limit the anomaly detection to major parts.
      \item Introduce further anomaly detector algorithms, for example One Class SVM or apply
            deep learning methods to obtain better kernels.
      \item Loss calculation can be refined to match better to the input and output images.
\end{itemize}

Before venturing to further improvements the opportunity is given to understand deeper how the
model works.
We have seen characteristic changes on the PCA / t-SNE representation.
This representation can be further tuned and extended with different approaches to visualize behavior
of the dataset de- and recomposition capability of the model.
Furthermore, analysis of the latent space is not yet exhausted, introduction of pairwise distance
calculation might reveal more information on the abstract representation of the images.
And last but not least, as further video footage is available we can obtain additional information
from real life videos that might deliver further information and guidance how the problem could
be tackled by machine learning approach.